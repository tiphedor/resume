
\documentclass[11pt]{article}

\usepackage[margin=1cm]{geometry}
\usepackage{titlesec}
\usepackage{setspace}
\usepackage{pifont}
\usepackage{graphicx}
\usepackage[english]{babel}
\usepackage{csquotes}
\usepackage{enumitem}
\usepackage{hyperref}

\setlength{\parindent}{0pt}
\setlength{\parskip}{0pt}
\setlist[itemize]{noitemsep}


\pagenumbering{gobble}

\titleformat{\section}{\large\bfseries}{\thesection.}{1em}{}
\titleformat{\subsection}{\normalsize\bfseries}{\thesubsection.}{1em}{}

% Section for experience
\newcommand{\experience}[4]{%
  \vspace{5pt}
  \textbf{#2} \hfill \textbf{#3} \\
  \textit{#1} \hfill \textit{#4} \\
  \vspace{-10pt}
}

% Bullet points for tasks and achievements
\newcommand{\bulletpoint}[1]{%
  \begin{itemize}
    \setlength\itemsep{-3pt}
    \setlength{\topsep}{0pt}
    \item #1
  \end{itemize}
}

\newcommand{\cvsection}[1]{%
  \section*{#1}
    \vspace{-10pt}
    \hrule
    \vspace{10pt}
}

\begin{document}

% Your header
\textbf{\Huge Martin STEFFEN} \textit{(il / lui)}

\large{\href{mailto:mail@tiph.io}{mail@tiph.io} \ding{117} 06 23 88 36 13 \ding{117} \texttt{github.com/tiphedor} \ding{117} \texttt{tiph.io}}

\vspace{10pt}
\hrule
\vspace{15pt}


Développeur Fullstack avec plus de cinq ans d'expérience, je suis passionné et polyvalent. Fort d'une grande capacité d'adaptation, je suis reconnu pour mon expertise en développement front-end et back-end, ainsi qu'en cloud, le tout dans des domaines d'application variés.

\cvsection{Expériences professionnelles}

\experience{Développeur Fullstack Senior}{Mindflow}{Fév 2023 - Fév 2025}{Full-remote}


Mindflow est une start-up proposant une solution SaaS d'automatisation no-code destinée au monde de la cybersécurité.

\begin{itemize}
    \item En tant que développeur fullstack, j'étais responsable de la réalisation complète des fonctionnalités de la plateforme
    \begin{itemize}
        \item Collaboration avec les équipes produit et les clients sur la conception et les améliorations
        \item Écriture du code de nouvelles fonctionnalités, du front au back, y compris les tests automatiques et manuels
        \item Intervention sur l'infrastructure cloud pour les nouveaux besoins, la résolution d'incidents et la limitation des coûts
    \end{itemize}
    \item J'étais fortement impliqué dans les processus d'onboarding et dans l'accompagnement de la montée en compétences des autres développeurs (pair-programming, code review, mentorat, etc.)
    \item J'ai été responsable de la migration de la codebase vers un monorepo, un chantier sur plusieurs mois, qui a permis à mon équipe de gagner en productivité et en vélocité
    \item \textbf{Stack technique}: TypeScript, React, GraphQL, AWS, CDK.
\end{itemize}

\experience{Développeur Front}{Energisme}{Fév 2022 - Fév 2023}{Full-remote}

Energisme gère une flotte d'appareils IoT dédiés à la mesure énergétique, dans une optique de sobriété et de réduction de l'empreinte carbone des entreprises et collectivités locales.

\begin{itemize}
    \item J'ai participé à la création from scratch d'un écosystème applicatif de collecte, visualisation et exploitation de données issues des mesures.
    \item Forte problématique de réutilisation de code et de déploiement différencié
    \item \textbf{Stack technique}: TypeScript, React, GraphQL, Azure, Java.
\end{itemize}

\experience{Développeur Front}{Bedrock Streaming}{Mar 2020 - Fév 2022}{Lyon, France}

Bedrock propose en marque blanche une plateforme de streaming à des chaînes de télévision et plateformes de SVOD / AVOD

\begin{itemize}
    \item Je travaillais à la création des applications pour télévisions connectées (Samsung Tizen, LG WebOs, etc.), et leurs problématiques spécifiques de performances et de composantes propriétaires
    \item Gestion de l'infrastructure cloud pour répondre aux nouveaux besoins, résoudre les incidents et optimiser les coûts
    \item J'ai été à l'origine d'un projet de R\&D qui consistait à contrôler à distance le parc de TV, initialement pour faciliter le travail lors de la pandémie
    \item \textbf{Stack technique}: JavaScript, React, Terraform, AWS.
\end{itemize}


\experience{Développeur Backend}{Michelin}{Jui 2019 - Fév 2020}{Lyon, France}

Michelin opère une centaine de sites commerciaux, pour ses clients particuliers et professionnels, dans de nombreuses langues et au travers de ses nombreuses marques.

\begin{itemize}
    \item Je participais au fonctionnement de cette usine à sites, en intervenant sur la partie back, notamment via des évolutions du CMS Apostrophe.
    \item Nous étions une équipe multinationale et multiculturelle travaillant presque exclusivement en anglais
    \item \textbf{Stack technique}: Node.js, MongoDB, Apostrophe CMS.
\end{itemize}

\experience{Stagiaire DevOps}{Zenika}{Sep 2018 - Fév 2019}{Lyon, France}
\vspace{-10pt}
\begin{itemize}
  \item Création d'une stack de déploiement réutilisable, allant du dépôt Git à l'application en production. L'idée était de proposer une solution clés en mains permettant de satisfaire tous les besoins DevOps d'une entreprise avec un déploiement en 5 minutes.
\end{itemize}

\experience{Webmaster, développeur}{MesBilles.fr}{2016 - Actuel}{Full-remote}
\vspace{-10pt}
\begin{itemize}
    \item Développeur et mainteneur d'un site de e-commerce basé sur Prestashop
    \begin{itemize}
      \item Développement en PHP de nouveaux modules et adaptation de modules existants
      \item Travail sur le SEO, l'accessibilité et la rapidité du site
    \end{itemize}
    \item Migration complète en 2022 sur Shopify
\end{itemize}

\cvsection{Éducation}

\experience{Formation au code en peer-learning}{42 Paris}{2019}{}

\experience{Licence Mathématiques et informatique}{Polytech Lyon}{2016}{}

\cvsection{Partage de Connaissances}

\begin{itemize}
    \item \textbf{Formations:} En parallèle de mes missions en clientèle, j'ai donné à plusieurs reprises des formations sur JavaScript, pour débutants et confirmés, à des publics variés, allant de jeunes en formation à des équipes d'entreprises.
    \item \textbf{Présentations:} À de nombreuses reprises, j'ai eu le plaisir de pouvoir partager mes réalisations devant des publics, notamment lors de conférences internes à Zenika, mais également lors de meetups, par exemple avec le CNCF.
\end{itemize}



\cvsection{Compétences, centres d'intérêts}

\begin{itemize}
  \item \textbf{Langues:} Français (natif), Anglais (bilingue)
  \item \textbf{Compétences diverses:} Culture et passion pour l'informatique, apprentissage autonome, dactylographie (120+ mpm), pédagogie, méthodes agiles
  \item \textbf{Centres d'intérêts} Électronique, impression 3D, urbanisme et systèmes ferroviaires, cyclisme et randonnée.
\end{itemize}

\end{document}
